\documentclass[11pt]{article}
\usepackage{monpaquet}
\usepackage{jupyter}


\begin{document}
%\tableofcontents
\pagestyle{fancy}
\cfoot{\lpgdg}
\chead{\textsc{Sp�cialit� NSI}}
\lhead{Premi�re G�n�rale}
%\chead{Nom : ..............................................................}
\rhead{\annee}%indiquer l'ann�e scolaire


\begin{center}
\begin{Large}\textbf{\textsc{Rappels de seconde sur les algorithmes\\ et\\ le langage Python}}\end{Large}
\end{center}

\section{I - Affectations et variables}
\begin{definition*}
Un algorithme est une suite finie d'op�rations �l�mentaires, � appliquer
dans un ordre d�termin�, � des donn�es.
\end{definition*}

Dans un algorithme ou un programme Python :
\begin{itemize}
\item  on utilise des \textbf{variables}. Elles repr�sente des nombres ou d'autres objets (listes, cha�nes de caract�re, \ldots{}) ;
\item on utilise une lettre ou un mot pour les d�signer ; 
\item on modifie leur valeur lors d'\textbf{affectations}.
\end{itemize}

L'affectation du nombre $3$ � la variable $a$ se note $a \aff 3$ en langage algorithmique et $a = 3$ en Python.
\end{document}